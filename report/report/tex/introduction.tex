\section{Introduction}
The construction industry faces challenges of worker safety, stagnant productivity due to an aging workforce and shortage of skilled labor, and inconsistent quality. Contrary to other industries facing similar issues, autonomous robotics has seen limited adoption by the construction industry as a solution to these challenges \cite{carra2018robotics,delgado2019robotics}. Robots excel in environments such as assembly lines due to their highly controlled and predictable nature. Construction environments, however, tend to be dynamic and unpredictable. Indeed, Carra et al. \cite{carra2018robotics} identified that one of the primary hurdles in the way of autonomous robotics being adopted by the construction industry is this unpredictability.

Though not wide-spread, there has been some exploration of the use of autonomous robots in construction. These robots tend to be large and/or immobile and designed for tasks involving heavy lifting, such as foundation laying and wall construction.\footnote{MULE135 and SAM100 by Construction Robotics (\href{https://www.construction-robotics.com}{https://www.construction-robotics.com}) and Hadrian X by FBR (\href{https://www.fbr.com.au/view/hadrian-x}{https://www.fbr.com.au/view/hadrian-x}) are examples of such robots.} However, large robots are not well suited to performing fine-grained tasks that must occur inside an assembled structure. Smaller construction robots capable of navigating throughout a building or construction site would augment the work of their larger counterparts by performing tasks impractical or impossible for large robots. However, little research and development has gone into such small-scale autonomous construction robots. The small construction robots that do exist tend to require remote control by an operator or specialize in performing exactly one task.

Thus, the goal of this project is to develop a proof of concept for a mobile, autonomous, general-purpose robotics platform capable of performing fine-grained tasks. There are four design tenants adhered to in pursuit of this goal:
\begin{enumerate}
    \item The platform must be autonomous. This will allow it to complete its tasks alongside human coworkers or outside of standard work site hours with minimal or no supervision.
    \item The platform must be mobile. This will allow it to perform tasks distributed throughout the work site.
    \item The platform must be small enough to fit through standard doorways. This will allow it to navigate throughout the work site.
    \item The platform must be flexible and extensible. This will allow it to adapt to and work in as wide a range of situations as possible.
\end{enumerate}

The Robot Operating System (ROS) \cite{ros_melodic} is an actively developed and well-supported software framework designed to facilitate all aspects of robotic systems. ROS provides the necessary structure to manage any number of subsystems and their communication. ROS was chosen as the foundation for the robotics platform presented by this project due to the wealth of existing support for the various subsystems required by it, including motor control, navigation, task planning, and sensor data collection.

Painting a wall with a conventional paint roller was selected as a representative task for demonstrating the feasibility of the robot. This task uses simple materials and tools, has clearly defined and verifiable success criteria, and requires a simple set of actions to complete. Simulation, instead of physical hardware, was used for this stage of development. Simulation is a much more rapid and cost effective means of development and testing before spending time and resources on physical robot components and a proper physical testing environment.

The work done in this project is intended to serve as a foundation for further development and is the first step toward developing a general-purpose mobile construction robot. While only a single task is demonstrated by this proof of concept, no aspect of the simulated robot is designed such that it prohibits extension of the robot's functionality to new tasks.

\subsection{Related Work}
Keerthanaa et al. \cite{keerthanaa2013automatic} designed a small mobile robot to spray paint vertical surfaces. The robot consists of a mobile base with four unpowered wheels and a vertically-oriented pulley system used to position a spray paint nozzle. The robot's painting operation is initiated by an operator and detects the limits of the wall with IR sensors. Singh et al. \cite{singh2018arduino} also designed a small mobile painting robot, however theirs is mounted to a base with four powered omni wheels and has a manipulator arm consisting of two links. The robot is equipped with a series of ultrasonic sensors on the base for navigation and on the arm for wall detection. Both robots were designed to be simple, inexpensive, and single-purpose. Nigl et al. \cite{nigl2013structure} presented an autonomous robot capable of traversing and reconfiguring a three dimensional truss structure. The robot consists of a hinge connecting two grippers that the robot uses to attach itself to and manipulate trusses. Combining motions of the grippers and hinge allows the robot to navigate between a connected series of trusses. This work explores the possibility of using an automated robot to construct or repair structures but requires the structures to adhere to a specific truss-based design.

Some work has been done on the use of multi-agent systems of autonomous robots in construction. In these approaches, multiple relatively simplistic robots coordinate and interact to perform tasks of a larger scale than their size and complexity would typically allow for. Werfel, Petersen, and Nagpal \cite{werfel2014designing} developed a system in which multiple small robots would arrange bricks to form a specified three dimensional structure. They were able to implement and test their design with three physical robots. Parker, Zhang, and Kube \cite{parker2003blind} developed a model for multiple robots clearing an area of debris. Their technique is based on an ant behavior known as ``blind bulldozing'' in which each ant plows in a straight line until a resistance threshold is encountered.

\subsection{Report Outline}
Section \ref{sec:architecture} details the robot's chassis, the software components that drive the robot's actions, and the ontology developed within this project. Section \ref{sec:interaction} describes the prototype user interface developed for use with the robot and explains the primary means of initiating the robot's actions. Section \ref{sec:results} details the experimental setup and discusses the results of the test simulations. Section \ref{sec:conclusion} discusses conclusions and the next steps to take in the overarching effort of the robot's development.