\section{Conclusion} \label{sec:conclusion}
This paper presents a functional proof of concept for a small, mobile, autonomous construction robotics platform. The platform is based on ROS and uses well known and well supported ROS-based packages to manage the task planning, arm motion control, and navigation of the robot. Though the only task demonstrated was wall painting, nothing in the design of the platform prohibits it from being expanded to work for arbitrary tasks. The use of a general purpose task planning framework, such as SkiROS, ensures this. The painting and navigation skills were shown to be able to interact with one another and allow for complex, dynamic behavior. However, there is still much work to be done and room for growth. The simulations show that the concept of such a robot is nevertheless feasible.

\subsection{Future Work}
The most serious issues with the platform, as designed thus far, are with some of the algorithms chosen. In particular, the configured navigation components have been shown to have difficulty planning paths that successfully direct the robot to its destination once in close proximity to the destination, even with holonomic motion. The TFD algorithm has shown irregular performance when planning painting tasks for a small number of entities. These issues are a high priority and should be resolved by either correcting the configuration of the algorithms or replacing the algorithms altogether.

The development of the robot should be continued using physical hardware. While software simulation is beneficial for rapid prototyping, it is a radically different environment from hardware in the physical world. The move to physical hardware will bring with it many changes. First, the size of the robot will need to be updated. The simulated YouBot is rather small -- the arm only vertically reaches \SI{0.655}{\meter} fully extended. While this has been sufficient for testing the general concepts motivating this project, it limits the robot's practical effectiveness. Second, a manipulator arm with more or less than the $5$ degrees of freedom tested in this project will greatly impact the sorts of actions that can be performed. Finally, the drive system (i.e. holonomic vs. non-holonomic) will dictate which navigation algorithms are viable.

Once the hardware to be used is more well understood, a system will need to be devised that allows the robot to utilize a wide variety of tools and equipment. This could mean grasping and manipulating tools with a gripper or a modular system that allows the robot to swap tool attachments on its arm. Additionally, the sensor suite on the robot will need to be expanded. The single front-mounted laser rangefinder present on the simulated YouBot is sufficient only for basic navigation. A robot working in a construction zone with dangerous tools and materials requires much greater situational awareness than represented here. Examples include adding more rangefinder devices with greater coverage area around the robot and one or more cameras for object recognition.

Building Information Modeling (BIM) is a process used in the construction industry for representing and managing construction requirements. Many popular construction planning software tools output files containing BIM data. This information has some conceptual similarities to languages and formats used to encode information in computer science, such as OWL and PDDL. It should be possible to bridge this gap and convert from one domain to the other. An application (such as the Scene Generator) could take BIM data as input and translate it into a format that can be used by planning algorithms. Incorporating BIM into the operation of the robot in this way could help to greatly increase both the value provided by the platform and its adoption rate by the construction industry.