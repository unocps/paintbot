\begin{abstract}
Utilizing autonomous robots to perform repetitive and labor-intensive tasks in the construction industry is one of the most promising directions to explore in order to enhance productivity, safety/health, and quality. Unlike many other industries, construction products, such as buildings, roads, bridges, are often designed to be built once and contain many unique parts. Robots must have construction-related knowledge and skills in order to generate task plans capable of dealing with the unique and highly dynamic work environment of typical construction sites. However, autonomous and flexible behavior is currently impossible due to a lack of construction knowledge base compatible with robotics. To overcome this bottleneck, this study proposes the establishment and utilization of such a construction knowledge base for use in generating autonomous behavior in robots. Specifically, this study provides an implementation of a small, mobile, autonomous robotics platform capable of performing fine-grained construction tasks in dynamic environments. Such tasks include painting, drilling screws, and transporting material and equipment. The platform is tested with a simulated robot based on the KUKA youBot tasked with painting walls in a room containing obstacles. The results of the simulations show that the proposed approach shows promise toward achieving autonomous operations of construction robots. Further development of this study will include implementing a more diverse set of skills, expanding the construction knowledge base, and tailoring localization, navigation planning, and task planning algorithms need to be tailored to the characteristics of the construction sites and the hardware tools used.

%Many industries, like manufacturing and services, successfully adopted robots to automate repetitive production processes. 
%For example, designs of rooms in a building can be slightly different or the rooms of the same shape may need to be constructed differently due to different spatiotemporal conditions in the building.
%The housing construction industry encounters conspicuous challenges of stagnant productivity, labors safety, an aging workforce, shortage of skilled labor, and inconsistent quality. The traditional human-driven construction methodology underperforms and fails to deliver optimum performance. Building information modeling (BIM) has been promoted to mitigate the process of construction design because of its data management capabilities from interdisciplinary work. However, using BIM, all components still rely on engineering work, including planning and design. To enhance the performance and productivity of on-site construction operation and labors safety, this research proposes the development of autonomous construction robotics collaborating with human workers in construction, and the robotic capability can reduce the need of interaction with multiple workers without predefined coordination scenarios and continuously changing working environment. Autonomous robots are underutilized in the construction industry. Current construction robots tend to be immobile and designed for large-scale tasks such as wall construction. This project develops the foundation for an autonomous, mobile robotics platform capable of performing fine-grained construction tasks. Examples of such tasks include painting, drilling screws, and transporting tools and equipment. The platform is successfully demonstrated using a simulated robot capable of autonomously painting walls.
\end{abstract}