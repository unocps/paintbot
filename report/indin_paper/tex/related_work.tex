\section{Related Work}\label{sec:related_work}

With an increasing need for robots, several applications have been introduced by the construction industry and academia. A semi-automated bricklaying robot developed by Construction Robotics demonstrated an improved speed of laying 300-400 bricks an hour compared to a human laying 60-75 bricks an hour \cite{bricklaying1}. DPR Construction~\cite{DPR1} introduced a robot that automatically draws drywall layouts according to building information models. With a goal of marking 1,000 linear feet per hour, the robot demonstrated a potential to reduce labor hours for expensive layout crews. Tybot and Brayman Construction~\cite{rebar1} developed a rebar-tying robot for bridge deck construction. After their first test resulting in 35\% savings in man-hours, they achieved a 40\% reduced rebar man-hours and 30 calendar days of schedule reduction in 2018 \cite{rebar2}. 

In academic studies, Jovanovic et al \cite{jovanovic2017robotic} proposed a method to use a robot to cut complex freeform shell panels made from foam materials. Lundeen et al \cite{lundeen2019autonomous} created a framework to adjust planned robot arm movement according to the sensory data collected during joint filling task. Keerthanaa et al. \cite{keerthanaa2013automatic} designed a small mobile robot to spray paint vertical surfaces. Its motion is initiated by an operator but completes autonomously. Singh et al. \cite{singh2018arduino} also designed a small mobile painting robot, however theirs is mounted to a base with four powered omni wheels and has a manipulator arm consisting of two links. Both painting robots were designed to be simple, inexpensive, and single-purpose. Nigl et al. \cite{nigl2013structure} presented an autonomous robot capable of traversing and reconfiguring a three dimensional truss structure. The robot consists of a hinge connecting two grippers that the robot uses to attach itself to and manipulate trusses. This work explores the possibility of using an automated robot to construct or repair structures but requires the structures adhere to a specific truss-based design. Some of our previous work~\cite{huang2018tradeoffs,huang2018skill} proposed robotic task designs.

Some work has been done on the use of multi-agent systems of autonomous robots in construction. In these approaches, multiple relatively simplistic robots coordinate and interact to perform tasks of a larger scale than their size and complexity would typically allow for. Werfel, Petersen, and Radhika \cite{werfel2014designing} developed a system in which multiple small robots would arrange bricks to form a specified three dimensional structure. They were able to implement and test their design with three physical robots. Parker, Zhang, and Kube \cite{parker2003blind} developed a model for multiple robots clearing an area of debris. The technique is based on an ant behavior known as ``blind bulldozing'' in which each ant plows in a straight direction until a certain resistance force is encountered.